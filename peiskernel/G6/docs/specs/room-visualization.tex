\documentclass{article}

\newcommand{\tn}{\bf}

\begin{document}
\section*{Specification of Interface to 3D visualisation of environment}

Any agent participating in the peis ecology can contribute with 
information of the physical contents of the rooms. All information 
is stored in the tuplespace, typically using the peis providing the 
information as the owner for the tuple. The visualisation tool continously 
queries the tuplespace for all relevant objects and visualizes them. In 
case there are tuples under more than one owner describing the contents 
of the room all these objects will be visualized. 

\noindent
The basic operation of the visualisation tool is as follows: 
On startup it connects to the tuplespace and subscribes to the
{\tn environment} tuple described below with wildcard as owner. 
After all other initialization routines it repeates the steps below.
\begin{enumerate}
\item Each {\tn environment} tuple is treated as a whitespace separated
list of named objects in the world. 
\item For each new named object {\tn name} a subscription is made to 
{\tn (name.property, owner)} where owner comes from the corresponding
{\sc environment} tuple and property are all applicable properties for the
object.
\item The properties of each object is queried using {\tt peisk\_getTuple}
and the object is drawn in the world. If an object is selected by 
the user additional tuple values (see below) is also subscribed to and
displayed.

\end{enumerate}

These are the tuples that are relevant for all objects.
\begin{description}
\item[environment] Contains a whitespace separated list of named objects identified in the room. Example ``chair0 table42 Emil''. All tuples
referenced below as {\tn name.property} is instanced as eg. {\tn chair0.property}.
\item[name.kind] As literal string describing the object and the model used for this object. Valid values include:
\begin{center} \tn chair, table, sofa, bookshelf, tv, loadspeakers, light, peoplebot, amigobot, pioneer, human, wall, floor, ceiling \end{center}
Example: {\tn chair0.kind = chair, Emil.kind = pioneer}
\item[name.position] A string containing the xyz posisition of the {\em base} of the object (lowermost point) represented as a string with three
printed real numbers giving the position in meters inside the global coordinate system. \\ Example: {\tn chair0.position = 2.0 -1.0 0.0}
\item[name.size] Gives the size of the unrotated object in meters. \\ Example: {\tn table42.size = 1.0 1.0 0.5}
\item[name.rotation] Gives the rotation of the object measured in {\em radians} as the three rotations {\em pan-tilt-yaw} and given in that order. \\
Example: {\tn chair0.rotation = 0.5 0.0 0.0}
\item[name.colour] A string giving the name of a colour that modulates the colour given for visualization. 
Example: {\tn chair0.colour = yellow}
\item[colourname.rgb] Gives the rgb values for colour strings (see above). \\ Example: {\tn yellow.rgb = 1.0 1.0 0.0}
\item[name.extra] Contains a whitespace separated list of additional properties to display for this object to the user as text strings. 
Example: {\tn \\ chair0.extra = ``'' \\ Emil.extra = status velocity \\ Emil.status = waiting \\ Emil.velocity = 0.0 0.0 0.0}

\end{description}

\end{document}
